% chktex-file  1
% chktex-file 17
% chktex-file 26

% ===============================
%  MACROS - Mathématiques
% ===============================

% ---------------------------
% PACKAGES ESSENTIELS
% ---------------------------
\usepackage[utf8]{inputenc}
\usepackage[T1]{fontenc}
\usepackage[french]{babel}
\usepackage{amsthm}
\usepackage{amsmath,amssymb,amsthm,mathtools}
\usepackage{bbm} % pour \mathbbm 1
\usepackage{mathrsfs} % pour \mathscr
\usepackage{geometry}
\usepackage{graphicx}
\usepackage{xcolor}
\usepackage{tcolorbox} % pour les encadrés
\usepackage{hyperref}
\usepackage{enumitem}
\usepackage{fancyhdr}
\usepackage{slashed}
\usepackage[version=4]{mhchem}
\usepackage{tikz}
\usepackage{titlesec}
\usepackage[compat=1.1.0]{tikz-feynman}
% ======== supprime les warnings LuaTeX ========
\makeatletter
\pgfkeys{
  /graph drawing/node distance/.code=\relax,
  /graph drawing/level distance/.code=\relax,
  /graph drawing/sibling distance/.code=\relax,
}
\makeatother
\tikzfeynmanset{warn luatex=false}

\tikzfeynmanset{
  every photon/.style={thick, color=yellow!90!black}
}


\usetikzlibrary{decorations.markings}

\usepackage{silence}
\WarningFilter{latexfont}{Font shape `T1/cmr/m/scit' undefined}


% -------------------------------
%  Ensembles Blackboard (\mathbb)
% -------------------------------
\newcommand{\bA}{\mathbb{A}}
\newcommand{\bB}{\mathbb{B}}
\newcommand{\bC}{\mathbb{C}}
\newcommand{\bD}{\mathbb{D}}
\newcommand{\bE}{\mathbb{E}}
\newcommand{\bF}{\mathbb{F}}
\newcommand{\bG}{\mathbb{G}}
\newcommand{\bH}{\mathbb{H}}
\newcommand{\bI}{\mathbb{I}}
\newcommand{\bJ}{\mathbb{J}}
\newcommand{\bK}{\mathbb{K}}
\newcommand{\bL}{\mathbb{L}}
\newcommand{\bM}{\mathbb{M}}
\newcommand{\bN}{\mathbb{N}}
\newcommand{\bO}{\mathbb{O}}
\newcommand{\bP}{\mathbb{P}}
\newcommand{\bQ}{\mathbb{Q}}
\newcommand{\bR}{\mathbb{R}}
\newcommand{\bS}{\mathbb{S}}
\newcommand{\bT}{\mathbb{T}}
\newcommand{\bU}{\mathbb{U}}
\newcommand{\bV}{\mathbb{V}}
\newcommand{\bW}{\mathbb{W}}
\newcommand{\bX}{\mathbb{X}}
\newcommand{\bY}{\mathbb{Y}}
\newcommand{\bZ}{\mathbb{Z}}

% -------------------------------
%  Ensembles Calligraphiques (\mathcal)
% -------------------------------
\newcommand{\cA}{\mathcal{A}}
\newcommand{\cB}{\mathcal{B}}
\newcommand{\cC}{\mathcal{C}}
\newcommand{\cD}{\mathcal{D}}
\newcommand{\cE}{\mathcal{E}}
\newcommand{\cF}{\mathcal{F}}
\newcommand{\cG}{\mathcal{G}}
\newcommand{\cH}{\mathcal{H}}
\newcommand{\cI}{\mathcal{I}}
\newcommand{\cJ}{\mathcal{J}}
\newcommand{\cK}{\mathcal{K}}
\newcommand{\cL}{\mathcal{L}}
\newcommand{\cM}{\mathcal{M}}
\newcommand{\cN}{\mathcal{N}}
\newcommand{\cO}{\mathcal{O}}
\newcommand{\cP}{\mathcal{P}}
\newcommand{\cQ}{\mathcal{Q}}
\newcommand{\cR}{\mathcal{R}}
\newcommand{\cS}{\mathcal{S}}
\newcommand{\cT}{\mathcal{T}}
\newcommand{\cU}{\mathcal{U}}
\newcommand{\cV}{\mathcal{V}}
\newcommand{\cW}{\mathcal{W}}
\newcommand{\cX}{\mathcal{X}}
\newcommand{\cY}{\mathcal{Y}}
\newcommand{\cZ}{\mathcal{Z}}

% -------------------------------
%  Ensembles Script (\mathscr)
% -------------------------------
\newcommand{\sA}{\mathscr{A}}
\newcommand{\sB}{\mathscr{B}}
\newcommand{\sC}{\mathscr{C}}
\newcommand{\sD}{\mathscr{D}}
\newcommand{\sE}{\mathscr{E}}
\newcommand{\sF}{\mathscr{F}}
\newcommand{\sG}{\mathscr{G}}
\newcommand{\sH}{\mathscr{H}}
\newcommand{\sI}{\mathscr{I}}
\newcommand{\sJ}{\mathscr{J}}
\newcommand{\sK}{\mathscr{K}}
\newcommand{\sL}{\mathscr{L}}
\newcommand{\sM}{\mathscr{M}}
\newcommand{\sN}{\mathscr{N}}
\newcommand{\sO}{\mathscr{O}}
\newcommand{\sP}{\mathscr{P}}
\newcommand{\sQ}{\mathscr{Q}}
\newcommand{\sR}{\mathscr{R}}
\newcommand{\sS}{\mathscr{S}}
\newcommand{\sT}{\mathscr{T}}
\newcommand{\sU}{\mathscr{U}}
\newcommand{\sV}{\mathscr{V}}
\newcommand{\sW}{\mathscr{W}}
\newcommand{\sX}{\mathscr{X}}
\newcommand{\sY}{\mathscr{Y}}
\newcommand{\sZ}{\mathscr{Z}}

% -------------------------------
%  Commandes mathématiques utiles
% -------------------------------

% Opérateurs
\DeclareMathOperator{\id}{id}
\DeclareMathOperator{\im}{Im}
\DeclareMathOperator{\rang}{rang}
\DeclareMathOperator{\tr}{Tr}
\DeclareMathOperator{\Sp}{Sp}
\DeclareMathOperator{\grad}{grad}
\DeclareMathOperator{\divg}{div}
\DeclareMathOperator{\rot}{rot}
\DeclareMathOperator{\Ker}{Ker}
\DeclareMathOperator{\cov}{Cov}
\DeclareMathOperator{\RC}{RC}
\DeclareMathOperator{\Jac}{Jac}
\DeclareMathOperator{\BC}{BC}
\DeclareMathOperator{\Mat}{Mat}
\DeclareMathOperator{\D}{D}
\DeclareMathOperator{\Hess}{Hess}

% Fonctions usuelles
% Fonctions arc
\newcommand{\Arccos}{\operatorname{Arccos}}
\newcommand{\Arcsin}{\operatorname{Arcsin}}
\newcommand{\Arctan}{\operatorname{Arctan}}

% Fonctions hyperboliques en notation française
\newcommand{\ch}{\operatorname{ch}} % cosinus hyperbolique
\newcommand{\sh}{\operatorname{sh}} % sinus hyperbolique
\newcommand{\thh}{\operatorname{th}} % tangente hyperbolique

% Fonctions inverses
\newcommand{\argch}{\operatorname{argch}} % arccosh
\newcommand{\argsh}{\operatorname{argsh}} % arcsinh
\newcommand{\argth}{\operatorname{argth}} % arctanh

% Logarithme complexe
\newcommand{\Arg}{\operatorname{Arg}}
\newcommand{\Log}{\operatorname{Log}}

% Différentiels et dérivées
\newcommand{\dd}{\mathrm{d}}
\newcommand{\deriv}[2]{\frac{\dd#1}{\dd#2}}
\newcommand{\pderiv}[2]{\frac{\partial#1}{\partial#2}}

% Limites, sommes, intégrales
\newcommand{\limn}{\lim_{n \to+\infty}}
\newcommand{\sumn}{\sum\limits_{n=0}^{\infty}}
\newcommand{\Sum}[3]{\sum_{#1=#2}^{#3}}
\newcommand{\intR}{\int_{-\infty}^{+\infty}}

% Notations usuelles
\newcommand{\abs}[1]{\left| #1 \right|}
\newcommand{\norm}[1]{\left\lVert#1 \right\rVert}
\newcommand{\normop}[1]{\lVert #1 \rVert_{\mathrm{op}}}
\newcommand{\paren}[1]{\left(#1 \right)}
\newcommand{\croch}[1]{\left[ #1 \right]}
\newcommand{\set}[1]{\left\{ #1 \right\}}

% Vecteurs et produits scalaires
\newcommand{\vect}[1]{\overrightarrow{#1}}
\newcommand{\scal}[2]{\left\langle#1, #2 \right\rangle}
\newcommand{\ps}[2]{\left\langle #1 \,\middle|\, #2 \right\rangle}
\newcommand{\pv}{\,_\wedge}

% Autres
\newcommand{\adh}[1]{\overline{#1}}
\newcommand{\tend}[2]{\xrightarrow[#1\to#2]{}}
\newcommand{\Res}[2]{\text{Res}(#1,#2)} %chktex 36
\newcommand{\vp}{\textbf{v.p.}}
\newcommand{\conv}{\mathop{\ast}\limits}
\newcommand{\trans}[1]{{#1}^\top}
\newcommand{\vtrans}[1]{\vec {#1}^\top}
\newcommand{\indic}{\mathbbm{1}}
\newcommand{\intseg}[2]{[\![#1, #2]\!]}
\newcommand{\dl}[1]{\underset{#1\to0}=}
\newcommand{\Matx}[2]{\Mat_{#1\gets#2}}
\newcommand{\cte}{\text{C}^\text{te}}
\newcommand{\vcte}{\vec\cte}
\newcommand{\eVc}{\text{ eV}\!\!\cdot\!\text{c}^{-2}}
\newcommand{\MeVc}{\text{ MeV}\!\!\cdot\!\text{c}^{-2}}
\newcommand{\GeVc}{\text{ GeV}\!\!\cdot\!\text{c}^{-2}}
\newcommand{\TeVc}{\text{ TeV}\!\!\cdot\!\text{c}^{-2}}
\newcommand{\GeVcp}{\text{ GeV}\!\!\cdot\!\text{c}^{-1}}
\newcommand{\MeV}{\text{ MeV}}
\newcommand{\GeV}{\text{ GeV}}
\newcommand{\MeVfm}{\text{ MeV}\!\!\cdot\!\text{fm}}
\newcommand{\tot}{\text{tot}}
\newcommand{\AN}{\mathscr{A\!.\,N\!.\;}}

% Redef
\let\oldexists\exists
\renewcommand{\exists}{\oldexists\,}

\makeatletter
\renewcommand*{\toclevel@part}{0}
\makeatother

% ----------------------------
\usepackage{enumitem}
\setlist[itemize,1]{label=\tiny\textbullet}
% ----------------------------

% ---------------------------
% ENVIRONNEMENTS MATHÉMATIQUES
% ---------------------------
% Mise en forme
\setcounter{secnumdepth}{4} % numérote jusqu’aux sous-sous-sections
\setcounter{tocdepth}{2}

\titleclass{\subsubsubsection}{straight}[\subsubsection]
\newcounter{subsubsubsection}[subsubsection]

\titleformat{\subsubsubsection}
  {\normalfont\normalsize\bfseries}
  {\thesubsubsubsection}
  {1em}
  {}

\titlespacing*{\subsubsubsection}
  {0pt}{3.25ex plus 1ex minus .2ex}{1.5ex plus .2ex}

\renewcommand{\thepart}{}
\renewcommand{\thesection}{\Roman{section}.}
\renewcommand{\thesubsection}{\arabic{subsection}.}
\renewcommand{\thesubsubsection}{\alph{subsubsection})} % chktex 9 chktex 10
\renewcommand{\thesubsubsubsection}{\roman{subsubsubsection}.}

\renewcommand{\thefootnote}{\fnsymbol{footnote}}

\theoremstyle{definition}
\newtheorem{definition}{Définition}[subsection]
\renewcommand{\thedefinition}{\thesection\thesubsection\arabic{definition}}
\newtheorem{lemme}{Lemme}[subsection]
\renewcommand{\thelemme}{\thesection\thesubsection\arabic{lemme}}
\providecommand*{\lemmeautorefname}{Lemme}
\newtheorem{theoreme}{Théorème}[subsection]
\renewcommand{\thetheoreme}{\thesection\thesubsection\arabic{theoreme}}
\newtheorem{propriete}{Propriété}[subsection]
\renewcommand{\thepropriete}{\thesection\thesubsection\arabic{propriete}}
\providecommand*{\proprieteautorefname}{Propriété}
\newtheorem{corollaire}{Corollaire}[subsection]
\renewcommand{\thecorollaire}{\thesection\thesubsection\arabic{corollaire}}
\newtheorem{exemple}{Exemple}[subsection]
\renewcommand{\theexemple}{\thesection\thesubsection\arabic{exemple}}
\newtheorem{notation}{Notation}[subsection]
\renewcommand{\thenotation}{\thesection\thesubsection\arabic{notation}}
\newtheorem{rappel}{Rappel}[subsection]
\renewcommand{\therappel}{\thesection\thesubsection\arabic{rappel}}

% Encadrés stylés pour définitions et théorèmes
\tcbset{
  mybox/.style={
    colback=#1!5,
    colframe=#1!80!black,
    boxrule=0.8pt,
    arc=4pt,
    left=6pt,right=6pt,top=6pt,bottom=6pt
  }
}

% Nouveau théorème nommé non numéroté
\newenvironment{namedtheorem}[1]{%
  \par\medskip
  \phantomsection % permet la référence hyperref
  \def\@currentlabelname{#1}% nom utilisé pour \nameref
}{%
  \par\medskip
}

% Commandes personnalisées
% D avec double slash oblique souscrit
\newcommand{\Dslash}{D_{\!\!\!\raisebox{1ex}{\scriptsize$\diagup\!\!\!\diagup$}}}


\newcommand{\defi}[2][\,]{%
  \begin{tcolorbox}[mybox=blue,title=\definition{#1}]
    #2
  \end{tcolorbox}
}
\newcommand{\ex}[2][\,]{%
  \begin{tcolorbox}[mybox=gray,title=\exemple{#1}]
    #2
  \end{tcolorbox}
}
\newcommand{\lem}[2][\,]{%
  \begin{tcolorbox}[mybox=orange,title=\lemme{#1}]
    #2
  \end{tcolorbox}
}
\newcommand{\Lem}[2][\,]{%
  \begin{tcolorbox}[mybox=orange,title=\textbf{Lemme #1}]
    #2
  \end{tcolorbox}
}
\newcommand{\nota}[2][\,]{%
  \begin{tcolorbox}[mybox=black,title=\notation{#1}]
    #2
  \end{tcolorbox}
}
\newcommand{\prop}[2][\,]{%
    \refstepcounter{propriete}%
    \begin{tcolorbox}[mybox=red,title=\textbf{Propriété~\thepropriete.}~{#1}]
    #2
    \end{tcolorbox}
}
\newcommand{\coro}[2][\,]{%
    \begin{tcolorbox}[colback=red!5!white, colframe=red!60!black, title=\corollaire{#1}]
    #2
    \end{tcolorbox}
}
\newcommand{\theo}[2][\,]{%
  \begin{tcolorbox}[mybox=purple,title=\theoreme{#1}]
    #2
  \end{tcolorbox}
}
\newcommand{\Theo}[2]{%
  \begin{tcolorbox}[mybox=purple,title=\textbf{Théorème #1}]
    #2
  \end{tcolorbox}
}
\newcommand{\Post}[2]{%
  \begin{tcolorbox}[mybox=purple,title=\textbf{Postulat #1}]
    #2
  \end{tcolorbox}
}
\newcommand{\TheoA}[2]{%
    \begin{namedtheorem}{#1}
        \begin{tcolorbox}[mybox=purple,title=\textbf{Théorème de \textsc{#1}}]
            #2
        \end{tcolorbox}
    \end{namedtheorem}
}
\newcommand{\DemoA}[2]{%
  \begin{tcolorbox}[mybox=pink,title=\textbf{$\Dslash$ \textsc{#1}}]
    #2
  \end{tcolorbox}
}
\newcommand{\Demo}[2][]{%
  \begin{tcolorbox}[mybox=pink,title=\textbf{$\Dslash$ #1}]
    #2
  \end{tcolorbox}
}
\newcommand{\rap}[2][\,]{
  \begin{tcolorbox}[mybox=violet,title=\rappel{#1}]
    #2
  \end{tcolorbox}
}
\newcommand{\rem}{\textbf{Remarque : }}
\renewcommand{\sectionmark}[1]{\markright{\thesection\ #1}}

% -------------------------------
%  Fin du fichier
% -------------------------------